% Generated by GrindEQ Word-to-LaTeX 
\documentclass{article} % use \documentstyle for old LaTeX compilers

\usepackage[english]{babel} % 'french', 'german', 'spanish', 'danish', etc.
\usepackage{amssymb}
\usepackage{amsmath}
\usepackage{txfonts}
\usepackage{mathdots}
\usepackage[classicReIm]{kpfonts}
\usepackage{graphicx}

% You can include more LaTeX packages here 


\begin{document}

%\selectlanguage{english} % remove comment delimiter ('%') and select language if required


\noindent 

\noindent 

\noindent 

\noindent 

\noindent \underbar{                                                             Datos no Estructurados}\textit{\underbar{ }}\underbar{Agosto 2020 }\textit{\underbar{  }}

\noindent 

\noindent 
\section{Datos no Estructurados}

\noindent 
\section{Belizario Mamani, Loja Mamani, Liz\'{a}rraga Paolo, Kenyi Chino}

\noindent 
\section{August 24, 2020}

\noindent 
\subsection{Abstract}

\noindent \textbf{}

\noindent Before talking about unstructured data, you need to understand what structured data is. When we talk about structured data we refer to the information that is usually found in most databases. They are text type files that are usually displayed in rows and columns with titles. It is data that can be easily ordered and processed by all data mining tools. We could see it as if it were a perfectly organized filing cabinet where everything is identified, labeled and easily accessible.

\noindent \underbar{}

\noindent 


\section{ Introduccion}

\noindent algunos casos de estos datos o informaci\'{o}n no est\'{a}n Los datos no estructurados, generalmente son datos binarios que no tienen estructura interna identificable. Es un conglomerado masivo y desorganizado de varios objetos que no tienen valor hasta que se identifican y almacenan de manera organizada.

\noindent 


\section{ RESUMEN}

\noindent 
\section{}

\noindent Antes de hablar de los datos no estructurados, es necesario comprender lo que son datos estructurados. Cuando hablamos de datos estructurados nos referimos a la informaci\'{o}n que se suele encontrar en la mayor\'{i}a de bases de datos. Son archivos de tipo texto que se suelen mostrar en filas y columnas con t\'{i}tulos. Son datos que pueden ser ordenados y procesados f\'{a}cilmente por todas las herramientas de miner\'{i}a de datos. Lo podr\'{i}amos ver como si fuese un archivador perfectamente organizado donde todo est\'{a} identificado, etiquetado y es de f\'{a}cil acceso.

\noindent 


\section{ TITULO}

\noindent 
\section{}

\noindent                 Datos no Estructurados

\noindent 


\section{ AUTORES}

\noindent 

\noindent Los autores de este trabajo son:

\noindent 

\begin{enumerate}
\item  Paolo Liz\'{a}rraga
\end{enumerate}

\noindent 
\section{\includegraphics*[width=1.95in, height=1.63in, keepaspectratio=false]{image1}}

\begin{enumerate}
\item \textbf{ }\includegraphics*[width=2.04in, height=1.91in, keepaspectratio=false]{image2}Anthony Belizario
\end{enumerate}

\noindent 

\noindent 

\noindent 

\begin{enumerate}
\item  Daniel Loja
\end{enumerate}

\noindent \includegraphics*[width=2.21in, height=2.06in, keepaspectratio=false]{image3}

\begin{enumerate}
\item  Kenyi chino
\end{enumerate}

\noindent 
\section{\includegraphics*[width=1.73in, height=2.27in, keepaspectratio=false]{image4}}

\noindent 
\section{}

\noindent 
\section{ V.DESARROLLO}

\noindent 

\noindent \textbf{¿QU\'{E} SON LOS DATOS NO ESTRUCTURADOS?}

\noindent \textbf{}

\noindent En su definici\'{o}n m\'{a}s b\'{a}sica, simplemente significa cualquier forma de datos que no encaja f\'{a}cilmente en un modelo relacional o un conjunto de tablas de base de datos desestructurado.

\noindent 

\noindent Los datos no estructurados, generalmente son datos binarios que no tienen estructura interna identificable. Es un conglomerado masivo y desorganizado de varios objetos que no tienen valor hasta que se identifican y almacenan de manera organizada.

\noindent 

\noindent Una vez que se organizan, los elementos que conforman su contenido pueden ser buscados y categorizados (al menos hasta cierto punto) para obtener informaci\'{o}n.

\noindent 

\noindent Aunque parezca incre\'{i}ble, la base de datos con informaci\'{o}n estructurada de una empresa, ni siquiera contiene la mitad de la informaci\'{o}n que hay disponible en la empresa lista para ser usada. El 80 \% de la informaci\'{o}n relevante para un negocio se origina en forma no estructurada, principalmente en formato texto.

\noindent 

\noindent 

\noindent 

\noindent \textbf{TIPOS DE DATOS NO-ESTRUCTURADOS:}

\noindent 

\noindent Entre los distintos tipos de datos no estructurados tenemos:

\begin{enumerate}
\item  Correos electr\'{o}nicos

\item  Archivos de procesador de texto como Word

\item  Archivos PDF

\item  Hojas de c\'{a}lculo como Excel.

\item  Im\'{a}genes digitales como formatos bmp, tiff

\item  V\'{i}deo como mp4, avi

\item  Audio como mp3

\item  Publicaciones en redes sociales.

\item  Presentaciones como PowerPoint
\end{enumerate}

\noindent Mirando esa lista, te podr\'{i}as preguntar qu\'{e} tienen en com\'{u}n estos archivos. Se trata de archivos que pueden ser almacenados y administrados sin que el sistema tenga necesidad de entender el formato del archivo. Al no estar organizado el contenido de estos archivos, estos datos suelen ser almacenados en carpetas locales en las redes de las empresas o en la nube como Dropbox, Google drive o SharePoint

\noindent 

\noindent \includegraphics*[width=2.27in, height=2.19in, keepaspectratio=false]{image5}

\noindent \textbf{FORMAS DE EXTRACCI\'{O}N DE DATOS NO ESTRUCTURADOS:}

\noindent Entre los m\'{e}todos de extracci\'{o}n de datos no estructurados tenemos:

\noindent \textbf{Web Scraping (Rascado de datos): }

\noindent Se podr\'{i}a definir como la t\'{e}cnica por la que un equipo de desarrolladores es capaz de rascar, escrapear o liberar datos de p\'{a}ginas web de gobiernos, instituciones p\'{u}blicas u organizaciones para acceder a datos privados o p\'{u}blicos que puedan ser publicados o distribuidos en formato abierto. El problema es que la mayor\'{i}a de los datos de inter\'{e}s est\'{a}n en formatos no reutilizables y poco transparentes como un PDF, por ejemplo.

\noindent \textbf{Extracci\'{o}n de datos con Python:}

\noindent En este ejemplo con librer\'{i}as como BeautifulShop que nos sirve para la extracci\'{o}n sencilla de datos concretos de una p\'{a}gina web en HTML sin excesiva programaci\'{o}n. Es lo que t\'{e}cnicamente recibe el nombre de parsear HTML. Una de las ventajas de esta biblioteca en Python es que todos los documentos salientes de la extracci\'{o}n de datos lo hacen en UTF-8, lo cual es bastante interesante porque el problema t\'{i}pico de las codificaciones queda totalmente resuelto

\noindent \textbf{Document Parsing (An\'{a}lisis de documentos):}

\noindent Se utiliza para analizar diferentes tipos de documentos como pdf, html, doc, presentaciones o im\'{a}genes. En alg\'{u}n momento es necesario que conserve el formato y la disposici\'{o}n del documento original, por ejemplo, en ocasiones, las estructuras originales de los p\'{a}rrafos, las estructuras de las tablas, los encabezados y subt\'{i}tulos y el mapeo de las secciones respectivas son importantes para una mejor precisi\'{o}n, por lo que debe conservarlos.

\noindent \textbf{Tokenizaci\'{o}n: }

\noindent Se trata de dividir el texto en varias oraciones, ya que ciertos procesos solo toman una oraci\'{o}n por vez. De manera similar, las oraciones deben convertirse en una secuencia de fichas para ciertos pasos.

\noindent \textbf{APLICACI\'{O}N DE LOS DATOS NO ESTRUCTURADOS, EJEMPLOS:}

\noindent 

\noindent \textbf{\includegraphics*[width=3.10in, height=2.82in, keepaspectratio=false]{image6}}

\noindent 

\noindent \textbf{\includegraphics*[width=3.01in, height=2.47in, keepaspectratio=false]{image7}}

\noindent 

\noindent 

\noindent . \textbf{\includegraphics*[width=3.01in, height=2.43in, keepaspectratio=false]{image8}}

\noindent 

\noindent 

\noindent \includegraphics*[width=3.25in, height=2.64in, keepaspectratio=false]{image9}

\noindent 

\noindent 

\noindent 

\noindent \textbf{     VI. CONCLUCIONES}

\noindent \textbf{}

\noindent No s\'{o}lo los datos estructurados pueden ser analizado.

\noindent 

\noindent Hay una gran disponibilidad de datos no estructurados.

\noindent 

\noindent Existen m\'{u}ltiples t\'{e}cnicas para analizar distintos tipos de datos no estructurados.

\noindent 

\noindent Hoy en d\'{i}a se est\'{a}n haciendo grandes avances en esta l\'{i}nea.

\noindent 


\section{ RECOMENDACIONES}

\noindent 
\section{}

\noindent 
\section{}

\noindent Para una correcta extracci\'{o}n de datos no estructurados recomendamos seguir 10 pautas:

\noindent 

\noindent Decide tu fuente de datos

\noindent Administra tu b\'{u}squeda de datos no estructurados

\noindent Elimina datos in\'{u}tiles

\noindent Prepara los datos para su almacenamiento

\noindent Decide la tecnolog\'{i}a para la pila y el almacenamiento de datos

\noindent Conserva los datos hasta que se almacenen

\noindent Recupera la informaci\'{o}n \'{u}til

\noindent Realizar la evaluaci\'{o}n de ontolog\'{i}a

\noindent Mantener un registro de estad\'{i}sticas

\noindent Analizar los datos

\noindent 
\section{}

\noindent 
\section{VIII. BIBLIOGRAFIA}

\noindent \textbf{}

 https://sommet.mx/blog/que-son-los-datos-no-estructurados

 https://www.kyoceradocumentsolutions.es/es/smarter-workspaces/insights-hub/articles/diferencia-entre-datos-estructurados-y-no-estructurados.html\#:$\mathrm{\sim}$:text=Los\%20datos\%20no\%20estructurados\%2C\%20generalmente,y\%20almacenan\%20de\%20manera\%20organizada.

 https://bbvaopen4u.com/es/actualidad/herramientas-de-extraccion-de-datos-para-principiantes-y-profesionales

 http://mbda.es/que-son-los-datos-no-estructurados/

 http://www.eco.unc.edu.ar/files/ief/workshops/2018/Galvez\_Extraccin\_y\_Anlisis\_de\_Datos\_No\_Estructurados\_\_Aplicaciones\_usando\_texto\_audio\_imgenes\_y\_video.pdf

 https://www.astera.com/es/topic/automated-data-extraction/automated-data-extraction-tools-for-faster-insights/

 Ontolog\'{i}a: Se ocupa de determinar dos aspectos importantes de las ontolog\'{i}as: calidad y correcci\'{o}n. ... Esto tambi\'{e}n ha presentado el desaf\'{i}o de decidir la idoneidad de una ontolog\'{i}a dada para los prop\'{o}sitos de uno en comparaci\'{o}n con otra ontolog\'{i}a en un dominio similar.


\end{document}

